\newpage
\section{Second Method - Burner Bypass Connection in CAB1}

If there is no connection point in CAB3, simply follow these steps.\\

Step 1: Ensure the MAGS System is in OFF mode, then turn off the main power from CAB1.\\

Step 2: Locate the burner bypass box (\emph{Figure \ref{fig:bypass_box}}).\\

Step 3: Open the CAB1 door to access the burner bypass connection point labelled CN1 (\emph{Figure \ref{fig:cab1_conn}}).\\

Step 4: Connect the burner bypass box to the connector and turn on the main power from CAB1 (\emph{Figure \ref{fig:cab1_connector}}).\\

Step 5: Remove the burner from the system and place it on a non-metallic surface. Ensure that no part of the burner is touching a metallic surface.\\

Step 6: Position the electrodes as shown on \emph{Figure \ref{fig:electrodes}}. Ensure that the tip of the electrodes are close to each other.\\

Step 7: Using the burner box, press the button labelled "Spark" or "Ignition". This will energize the electrodes and if they are positioned properly will cause an arc across the front of the nozzle. If there is a spark to the burner housing or no visible spark but you hear a buzz sound this means the electrodes need to be re-positioned.\\

Step 8: Repeat steps 6-7 until the spark is consistently in the right location.\\

Step 9: Disconnect the burner box and make place the jumper connector back.\\

Step 10: Replace the burner into the MAGS without moving the electrodes.\\

Step 11: Start the system normally.