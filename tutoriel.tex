%%%%%%%%%%%%%%%%% Tutoriel rapide %%%%%%%%%%%%%%%%%%%


% -- Type de section -- %
 Du plus gros au plus petit
    \section{}
    \subsection{}
    \subsubsection{}
    \paragraph{}
    la balise \paragraph{} n'apparaît pas dans la table des matière


% -- Formatage text -- %
    \emph{ceci a de l'emphase}
    \textit{ceci est en italique}
    \textbf{ceci est en gras}

    %Affichage de températures
    \degC pour celcius
    \degF pour fahrenheit

% -- Images -- %
   \begin{figure}[H]
   \centering
   \includegraphics[width=XXcm]{NOM.EXTENTION}
   \caption{NOM EN DESSOUS DE L'IMAGE} 
       \label{IDENTIFIANT_UNIQUE}
   \end{figure}

    %Options de placement
        H : Here, à la même place
        B : Bottom, au bas de la page
        P : Page, au haut de la page suivante
        T : Top, au haut de la page
    
    %Label
        Le label est un identifiant unique qui permet de 
        citer l'image dans le texte avec :
        
        \ref{label}
            Cite le numéro de la figure
            
        \pageref{label}
            Cite la page de l'image
 

% -- Formules mathématiques -- %
    On peut insérer des mathématiques dans et hors du texte.
    
    Dans le texte :
        La fonction $f(x) = x^2$ décrit une courbe quadratique...
    
    Hors texte :
        \begin{equation}
            f(x) = (x+1)^2 \\
            f(x) = x^2 + 2x + 1
        \end{equation}
        
    Hors texte :
    $$ \frac{x^3}{\sqrt{3}}  = \text{A}_i^{T/R} $$

    En hors texte, on peut faire des équations sur plusieurs lignes
    en utilisant \\ pour changer de ligne.
    
    Pour utiliser des symboles, simplement utiliser le \ :
        $ \lambda $
        $ \pi $
        ...

